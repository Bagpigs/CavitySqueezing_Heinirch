\section{Modification of dissipative couplings}
In the following chapters we observe the dependence of different squeezing criteria on the power P of the laser drive. We chose $\omega_z = 2 \pi \cross 7.09$ MHz, $\kappa = 2 \pi \cross 1.25$ MHz, an the coupling rate $\chi_{+,\text{fix}} = 2\pi \cross 0.15$Hz to be fixed. (\ref{eq:fou_coupling}) yields the following relation for $\eta$ and $\delta_+$
\begin{equation}
	\eta = \sqrt{\chi_+ \frac{(\delta_+^2+\kappa^2)}{\delta_+}}
\end{equation}
%Note, that this leads for $|\delta_+| \gg \kappa$ to approximately $P \propto \delta_+$. 
Specifically, we vary the cavity frequency $\omega_c$ which leads to changes in $\delta_+$ and correspondingly we vary the laser power P which leads to changes in $\eta$ (\ref{eq:fou_power_eta_prop}). The smallest Raman coupling we consider is $\eta = 2 \pi \cross 1.70$ kHz corresponding to the detuning $\delta_+ \approx - 2 \pi \cross 19.16$ MHz %old value was 18.71 when deltap was fixed and not xp
 and the highest is $\eta \approx 2 \pi \cross 2.94$ MHz corresponding to the detuning $\delta_+ \approx - 2 \pi \cross 57.77$ MHz. The values are chosen such that highest Raman coupling corresponds to three times the laser drive power of the smallest Raman coupling. 
 \\
 Regarding the EOM (\ref{eqom}) we see that this procedure will result in a change of only the variables $\chi_-, \gamma_\pm$. 
Thus, the ratio $\chi_+ / \gamma_+$ gets more favourable for a higher laser drive power.  